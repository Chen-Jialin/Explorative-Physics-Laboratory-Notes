\documentclass{assignment}
\ProjectInfos{研究型物理实验}{PHYS1703}{2020-2021学年第一学期}{Note-2-由四维矢量的洛伦兹变换\\推导多普勒效应}{2020. 9. 23(周三)}{陈稼霖}{45875852}
\begin{document}
\section*{要求}
由四维矢量$(\omega,k)$的洛伦兹变换推导多普勒效应.

\section{洛伦兹变换的引入}
从低速力学现象上归纳出来的旧时空观认为,惯性系$\Sigma$与相对于$\Sigma$以速率$v$沿$x$轴运动的惯性系$\Sigma'$中的位置坐标和时间遵循伽利略变换
\begin{align}
    x'=&x-vt,\\
    y'=&y,\\
    z'=&z,\\
    t'=&t.
\end{align}

理论上和实验上的矛盾使人们重新思考洛伦兹变换:
\begin{itemize}
    \item 理论上,由麦克斯韦方程组推导波动方程,最终解得平面波解,其中真空中光速为一常数. 在这里,没有给定一个特定的参考系,就有了为常数的光速.
    \item 起初认为这一光速是相对于某个特定参考系(例如,以太)而言的,则相对于这一参考系转动的地球上应能观测到这一差异,然而该假设被迈克尔逊-莫雷实验真伪.
\end{itemize}

因此,爱因斯坦提出狭义相对论的两条基本假设
\begin{itemize}
    \item[(1)] \textbf{相对性原理}:所有惯性参考系都是等价的. 物理规律对于所有惯性参考系都可表为相同形式. 即,无论通过力学现象,还是电磁现象,或其他现象,都无法察觉处所处参考系的任何“绝对运动”;
    \item[(2)] \textbf{光速不变原理}:真空中的光速相对于任何惯性系沿任一方向恒为$c$,并与光源运动无关.
\end{itemize}

根据这两条基本假设,从一个简单的例子,我们就可以推出洛伦兹变换. 考虑惯性系$\Sigma$及相对于$\Sigma$运动的另一惯性系$\Sigma'$,不失一般性,设$\Sigma'$相对于$\Sigma$运动的速度大小为$v$而方向沿$x$轴. 用$(\bm{x},t)=(x,y,z,t)$表示在惯性系$\Sigma$中发生的某一事件的空时坐标,用$(\bm{x}',t)=(x',y',z',t')$表示该事件在惯性系$\Sigma'$中对应的空时坐标.

假设有一物体相对于惯性系$\Sigma$做匀速运动,其运动方程应由$\bm{x}$与$t$的线性关系所描述,根据上述的狭义相对论的第一条基本假设——相对性原理,在惯性系$\Sigma'$上所观察到的该物体也应做匀速运动,从而其在惯性系$\Sigma'$中的运动方程应由$\bm{x}'$与$t'$中的线性关系描述,由此可知,从$(\bm{x},t)$到$(\bm{x}',t')$的变换必须为线性的,即,$x',y',z',t'$必须可表为$x,y,z,t$的线性组合.

假设在$t=0$时刻,两惯性系原点重合,且此时一个位于惯性系$\Sigma$原点的光源发光,对于光从光源发出这一事件(事件A),其在惯性系$\Sigma$和$\Sigma'$中的空时坐标均为$(0,0,0,0)$,对于某探测器接受到该光信号这一事件(事件B),设其在惯性系$\Sigma$和$\Sigma'$中的空时坐标分别为$(x,y,z,t)$和$(x',y',z',t')$. 根据上述的狭义相对论的第二条基本假设——光速不变性,有
\begin{align}
    x^2+y^2+z^2=&c^2t^2,\\
    x'^2+y'^2+z'^2=&c^2t'^2.
\end{align}
设
\begin{align}
    F_1(x,y,z,t)=&x^2+y^2+z^2-c^2t^2,\\
    F_2(x',y',z',t')=&x'^2+y'^2+z'^2-c^2t'^2.
\end{align}
因为$(x',y',z',t')$可表为$(x,y,z,t)$的线性组合,故$F_2$可进一步表为$(x,y,z,t)$的函数,$F_2(x,y,z,t)$.

我们可以任意摆放探测器,也就是说,我们可以任意调整$(x,y,z,t)$,但是无论怎么调整,只要事件A和事件B是以光来连接的,就恒有,$F_1(x,y,z,t)=0$,且$F_2(x,y,z,t)=0$. 当然两个事件也可以有除光以外的其他的连接方式(例如把事件A改成声波从声源发出,把事件B改成探测器探测到声波,例如把事件A改成飞船从基地出发,把事件B改成飞船到达目的地),此时则有$F_1(x,y,z,t)<0$,$F_2(x,y,z,t)<0$. 综合起来,有:\uline{$F_1(x,y,z,t)\leq 0$,\\$F_2(x,y,z,t)\leq 0$,且若$F_1(x,y,z,t)=0$(即,若两事件以光连接),则必有$F_2(x,y,z,t)=0$}.

由于$(x',y',z',t')$可表为$(x,y,z,t)$的线性组合,而又有以上划线的这条结论,所以$F_1(x,y,z,t)$和$F_2(x,y,z,t)$之间只相差一因子,即
\begin{align}
    x'^2+y'^2+z'^2-c^2t'^2=A(x^2+y^2+z^2-c^2t^2).
\end{align}
又因为$\Sigma$和$\Sigma'$这两个惯性系是等价的,反过来亦有关系
\begin{align}
    x^2+y^2+z^2-c^2t^2=A(x'^2+y'^2+z'^2-c^2t'^2).
\end{align}
从而有间隔不变性
\begin{align}
    \label{gap-const}
    x'^2+y'^2+z'^2-c^2t'^2=x^2+y^2+z^2-c^2t^2.
\end{align}

由于$\Sigma'$相对于$\Sigma$运动的速度沿$x$轴,故$y,z$在变换中不变,变换具有如下形式
\begin{align}
    \label{transform-1}x'=&a_{11}x+a_{12}ct,\\
    y'=&y,\\
    z'=&z,\\
    \label{transform-4}ct'=&a_{21}x+a_{22}ct.
\end{align}
由于$x$和$x'$的正向相同,故应取$a_{11}>0$;由于$t$和$t'$的正向相同,应取$a_{22}>0$,将变换的形式(式\eqref{transform-1}至式\eqref{transform-4})代入间隔不变性(式\eqref{gap-const})中得
\begin{align}
    (a_{11}x+a_{21}ct)^2+y^2+z^2-(a_{21}x+a_{22}ct)^2=x^2+y^2+z^2-c^2t^2.
\end{align}
展开后比较系数得
\begin{align}
    \label{1}a_{11}^2-a_{21}^2=&1,\\
    \label{2}a_{11}a_{12}-a_{21}a_{22}=&0,\\
    \label{3}a_{12}^2-a_{22}^2=&-1.
\end{align}
由式\eqref{1}和\eqref{3}得
\begin{align}
    \label{4}a_{11}=&\sqrt{1+a_{21}^2},\\
    \label{5}a_{22}=&\sqrt{1+a_{12}^2}.
\end{align}
在代入式\eqref{2}中得
\begin{align}
    \label{6}a_{12}=a_{21}.
\end{align}
这些系数与两个惯性系之间的相对运动速度有关,因而可以用$v$表出. 在$\Sigma$上观察,$\Sigma'$的原点$O'$以速率$v$沿$x$轴运动,其坐标为
\begin{align}
    x=vt.
\end{align}
而$O'$在$\Sigma'$上的坐标恒为$x'=0$,将其代入式\eqref{transform-1}中可得
\begin{align}
    \frac{a_{12}}{a_{11}}=-\frac{v}{c}.
\end{align}
从而由式\eqref{4}\eqref{5}\eqref{6}解得
\begin{align}
    a_{11}=&a_{22}=\frac{1}{\sqrt{1-\frac{v^2}{c^2}}}.\\
    _{12}=&a_{21}=\frac{-\frac{v}{c}}{\sqrt{1-\frac{v^2}{c^2}}}.
\end{align}
至此,我们得到了洛伦兹变换
\begin{align}
    x'=&\frac{x-vt}{\sqrt{1-\frac{v^2}{c^2}}},\\
    y'=&y,\\
    z'=&z,\\
    t'=&\frac{t-\frac{v}{c^2}x}{\sqrt{1-\frac{v^2}{c^2}}}.
\end{align}

\section{四维矢量的引入}
将上面的四个洛伦兹变换式写成线性代数形式:
\begin{align}
    \left(\begin{matrix}
        x'\\
        y'\\
        z'\\
        ict'\\
    \end{matrix}\right)=A\left(\begin{matrix}
        x\\
        y\\
        z\\
        ict
    \end{matrix}\right),
\end{align}
其中变换矩阵
\begin{align}
    A=\left(\begin{matrix}
        \gamma&0&0&i\beta\gamma\\
        0&1&0&0\\
        0&0&1&0\\
        -i\beta\gamma&0&0&\gamma
    \end{matrix}\right),
\end{align}
其中
\begin{align}
    \beta=&\frac{v}{c},\\
    \gamma=&\frac{1}{\sqrt{1-\frac{v^2}{c^2}}}.
\end{align}
可以验证,$A$是一个正交矩阵:
\begin{align}
    AA^T=I,
\end{align}
也就是说乘上这样一个矩阵相当于使矢量在四维空间中转动. 根据正交矩阵的性质,变换前后的两个矢量$(\bm{x},ict)$和\\$(\bm{x},ict')$模长不变.

\section{用四维矢量推导多普勒效应}

在惯性系$\Sigma$中,电磁波的相位为
\begin{align}
    \phi=\bm{k}\cdot\bm{x}-\omega t.
\end{align}
在惯性系$\Sigma'$中,电磁波的相位为
\begin{align}
    \phi'=\bm{k}'\cdot\bm{x}'-\omega't'.
\end{align}
假设在$t=0$时刻两个惯性系中观察到相位均为$0$,$\phi=\phi'=0$(事件A),事件A在两个惯性系中的空时坐标均为$(0,0,0,0)$,在惯性系$\Sigma$中经过电磁波的$n$个周期后,第$n$个波峰经过$\Sigma$原点,相位为$\phi=-2\pi n$(事件B),事件B在惯性系$\Sigma'$中的空时坐标为$(\bm{x}=0,t=2\pi n/\omega)$,通过洛伦兹变换可得事件$B$在惯性系$\Sigma'$中的空时坐标$(\bm{x}',t')=A(\bm{x},t)$,而相位仍为$\phi'=-2\pi n$,这是因为某个波峰经过原点是一个物理事件,而相位对应了经过原点的波峰的数目,不应随参考系的变换而发生改变:
\begin{align}
    \bm{k}\cdot\bm{x}-\omega t=\bm{k}'\cdot\bm{x}'-\omega't'
\end{align}
要满足上面这一点,波矢和频率的洛伦兹变换就需满足
\begin{align}
    \left(\begin{matrix}
        k_x'\\
        k_y'\\
        k_z'\\
        i\omega'/c
    \end{matrix}\right)=A^{-1}\left(\begin{matrix}
        k_x\\
        k_y\\
        k_z\\
        i\omega/c
    \end{matrix}\right)=A^T\left(\begin{matrix}
        k_x\\
        k_y\\
        k_z\\
        i\omega/c
    \end{matrix}\right).
\end{align}
从而有多普勒效应:
\begin{align}
    \omega'=\gamma(\omega-vk_1)=\gamma\left(\omega-v\frac{\omega}{c}\cos\theta\right)=\gamma\omega\left(1-\frac{v}{c}\cos\theta\right),
\end{align}
其中$\theta$为$\bm{k}$与$x$轴方向的夹角.

\section{讨论}
沿着激光传输方向运动的粒子的多普勒效应为
\begin{align}
    \omega'_{\parallel}=\gamma\omega\left(1-\frac{v}{c}\right)=\frac{\omega\left(1-\frac{v}{c}\right)}{\sqrt{1-\frac{v^2}{c^2}}}.
\end{align}
垂直于激光传输方向运动的粒子的多普勒效应为
\begin{align}
    \omega'_{\perp}=\gamma\omega=\frac{\omega}{\sqrt{1-\frac{v^2}{c^2}}}.
\end{align}
对于本实验中的气体粒子,其运动速率$v$远低于光速$c$,故
\begin{align}
    \omega'_{\perp}\approx\omega\left(1+\frac{1}{2}\frac{v^2}{c^2}\right)
\end{align}
与入射激光原始频率$\omega$的差距远远小于
\begin{align}
    \omega'_{\parallel}\approx\omega\left(1-\frac{v}{c}\right)
\end{align}
与入射激光原始频率$\omega$的差距,故总体上只需考虑沿着激光传输方向的速度造成的多普勒效应:
\begin{align}
    \omega'=\omega\left(1-\frac{v}{c}\right).
\end{align}

\nocite{*}
\bibliographystyle{plain}
\bibliography{Note-2}
\end{document}